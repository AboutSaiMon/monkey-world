\documentclass[a4paper,10pt]{article}
\usepackage[italian]{babel}
\usepackage[utf8x]{inputenc}

\title{Monkey World}
\author{
    Carmine Dodaro, Simone Spaccarotella \\
    \texttt{\{carminedodaro, spa.simone\}@gmail.com}
}

\begin{document}

    \maketitle

    \begin{abstract}
	Monkey World è un framework a singolo agente, per la simulazione di sistemi intelligenti. L'agente
	in questione è una scimmietta che può muoversi lungo un ambiente monodimensionale. Il suo obbiettivo è quello di rubare un
	casco di banane appeso per aria e ritornare a cuccia. Essa ha a disposizione una cassa che può spostare liberamente, sulla quale dovrà salire per poter
	raggiungere il suo scopo. In Monkey World è stato implementato un comportamento che permettesse alla scimmietta di portare a termine
	la sua missione in maniera razionale.
    \end{abstract}

    \section{Introduzione}
	Abbiamo un ambiente monodimensionale, composto da un nastro di dieci celle adiacenti, più una cella extra che
	rappresenta la cuccia della scimmietta. Questa cuccia può essere adiacente ad una sola cella del nastro e la sua
	posizione viene inizialmente settata da input, insieme alla cassa e al casco di banane.
	
	Le azioni che può compiere l'agente scimmietta sono: uscire dalla cuccia, tornare a cuccia,
	spostarsi sul nastro verso sinistra/destra, spostare la cassa verso sinistra/destra, salire sulla cassa,
	rubare il casco di banane, scendere dalla cassa. Per spostare la cassa, o per salirci sopra, la scimmietta deve trovarsi
	nella sua stessa locazione. Lo scopo è quello di portare la cassa sotto il casco di banane, salirci sopra, rubare il casco, 
	riportare la cassa al suo posto e ritornare a cuccia.
	
	Sono previsti tre tipologie di ambiente. Un ambiente statico, in cui la scimmietta ha tutto il tempo di pianificare le sue mosse
	ed agire. Un ambiente dinamico, dove in questo caso il casco di banane si muove in una posizione random, allo scadere di un intervallo di tempo fissato
	da input. Infine un ambiente ancora dinamico, ma questa volta il casco di banane viene mosso dall'utente via interfaccia grafica, mediante
	l'utilizzo del mouse. In tutti e tre i casi, l'ambiente è deterministico, episodico, discreto, a singolo agente e completamente osservabile,
	il che significa che l'agente ha piena percezione della posizione di tutti gli oggetti in ogni momento.
	
    \section{Ambiente Statico}
	In questo tipo di ambiente, nulla cambia con il passare del tempo. Questo permette alla scimmietta di pianificare le sue mosse in
	maniera del tutto off-line, ed eseguirle in un secondo momento.
	
	La pianificazione è stata fatta mediante l'utilizzo di \emph{DLV-K}, un frontend di dlv che implementa K, un linguaggio di pianificazione
	basato sulla logica dei predicati, per la rappresentazione di conoscenza incompleta.
	
	L'idea è la seguente. Viene calcolato solo metà piano, ovvero la parte in cui la scimmia esce dalla cuccia e ruba il casco di banane.
	Questa metà è stata suddivisa in ulteriori due sotto pianificazioni, una in cui la scimmietta deve raggiungere
	la cassa, ed un'altra in cui deve spostare la cassa fin sotto il casco di banane. In questo modo è stato possibile tagliare enormemente
	lo spazio di ricerca, diminuendo di conseguenza la complessità computazionale (in termini di tempo) del calcolo dell'intero piano.
	
	Per l'interfacciamento con \emph{DLV-K} è stata utilizzata la classe \emph{ProcessBuilder} messa a disposizione dalla JDK.
	Grazie a questa classe è stato possibile lanciare DLV come un sotto processo.
	Dopodiché è bastato agganciarsi allo stream di output, parserizzare
	la stringa e creare una struttura dati da dare in pasto all'agente, sotto forma di lista delle azioni da compiere.
	
	Come specificato in precedenza, il piano calcolato è solo metà. Questo perché la sequenza di azioni
	da compiere per riportare la cassa al suo posto e ritornare a cuccia è perfettamente speculare al piano già calcolato, quindi
	sarebbe stato inutile sprecare del tempo per calcolare qualcosa che è già stato calcolato in precedenza. E' bastato
	invertire la sequenza delle azioni, dove ogni azione era l'opposta di quella effettuata nella prima fase.
	
	Ad esempio: se nella prima fase la scimmia si sposta a destra per due volte, e poi sposta la cassa a sinistra per tre volte,
	nella seconda fase, invertendo la sequenza, avremo che la scimmia dovrà
	prima spostare la cassa per tre volte, ma questa volta a destra, e poi dovrà spostarsi per due volte, ma questa volta a sinistra.
	
    \section{Ambiente dinamico}

    \section{Ambiente dinamico, con intervento dell'utente}

\end{document}